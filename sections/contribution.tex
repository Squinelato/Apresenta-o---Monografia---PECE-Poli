\section{Prováveis Contribuições} % Seções são adicionadas para organizar sua apresentação em blocos discretos, todas as seções e subseções são automaticamente exibidas no índice como uma visão geral da apresentação, mas NÃO são exibidas como slides separados.

%------------------------------------------------

\begin{frame}
	\frametitle{Prováveis Contribuições}
    No contexto do mercado acionário brasileiro, almeja-se:
    \begin{enumerate}
        \item Criar uma base de notícias anotados
        \item Contribuir para o estado-da-arte quanto à AS por meio de AH e LLMs
    \end{enumerate}
	
\end{frame}

%------------------------------------------------
